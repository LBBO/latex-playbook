\documentclass[10pt]{article}

\usepackage[utf8]{inputenc}
\usepackage[a4paper, margin=1in]{geometry}
\usepackage[ngerman]{babel}
\usepackage{setspace}
\usepackage[dvipsnames]{xcolor}
\usepackage{colortbl} % fill table cells with color
\usepackage{float} % place a table where you want it and not where LaTeX wants it because of free space
\usepackage{lipsum}
\usepackage[sfdefault]{roboto}
\usepackage[T1]{fontenc}
\usepackage{soul} % underline links
\usepackage[hidelinks, linkcolor=black, urlcolor=blue, colorlinks=true]{hyperref} % create nice looking links (blue)

\newcommand{\PlayTitle}{Lorem ipsum Play}
\newcommand{\GroupName}{Lorem ipsum Group}
\hypersetup{
bookmarks=true, % set bookmarks for every section, subsection etc. in the PDF file
bookmarksnumbered=true,
pdftitle={\PlayTitle}, % set the title and author in the PDF meta info
pdfauthor={\GroupName},
}

\definecolor{AppearingCharacters}{RGB}{67, 67, 67}
\definecolor{SignatureLineColor}{RGB}{136, 136, 136}
\definecolor{TableColorAppearance}{RGB}{153, 153, 153}
\definecolor{TableColorSemiAppearance}{RGB}{204, 204, 204}

\newcommand{\character}[1]{\par\textbf{\MakeUppercase{#1}}\refstepcounter{#1}\label{#1}}
\newcommand{\InlineStageDirection}[1]{\textit{(#1)}}
\def\BlockStageDirection#1 \par{
	\textit{#1}\par % \par turns #1 into it's own paragraph with usual formatting
}
\newcommand{\scene}[2]{{
	\setlength{\parskip}{20pt}
	{\fontsize{16}{16}\textbf{\MakeUppercase{#1}}}\\
	\textcolor{AppearingCharacters}{\fontsize{14}{14}\selectfont#2}
	\par % \par turns #1 into it's own paragraph with usual formatting
}}

\newcommand{\TitlePageLine}[3]{\fontsize{#1}{#2}\selectfont #3 \par}
\newcommand{\TitlePageLineBold}[3]{\fontsize{#1}{#2}\selectfont \textbf{#3} \par}

% Force empty line between paragraphs
\setlength{\parskip}{1em}

\onehalfspacing

\begin{document}


\begin{titlepage}
\begin{center}
\TitlePageLineBold{18pt}{18pt}{Dieser Text gehört:} 
\vspace{1.25cm}
\textcolor{SignatureLineColor}{\rule{1\textwidth}{0.8pt}}

\TitlePageLine{18pt}{18pt}{Mitglied des} 
\TitlePageLineBold{36pt}{36pt}{\GroupName,}
\TitlePageLine{18pt}{18pt}{der am} 
\TitlePageLineBold{36pt}{36pt}{10. + 11. Januar 2020}
\TitlePageLine{18pt}{18pt}{im} 
\TitlePageLineBold{36pt}{36pt}{MZ der RUB}
\TitlePageLine{24pt}{36pt}{das Stück “\PlayTitle” von Heinrich von Kleist aufführt.}
\TitlePageLineBold{36pt}{36pt}{Eintritt frei!}
\vspace{2.5cm}
\TitlePageLine{18pt}{27pt}{Falls Sie diesen Text finden, würden wir uns sehr freuen, wenn Sie ihn im Musischen Zentrum der RUB abgeben \mbox{} oder eine Mail an \setuldepth{info}\href{mailto:info@chaostrub.de}{\ul{info@chaostrub.de}} schreiben würden.}
\end{center}
\end{titlepage}

\begin{titlepage}
\begin{flushleft}
\TitlePageLineBold{26pt}{26pt}{\setuldepth{Der}\ul\PlayTitle}
\vspace{0.5cm}

%%%%%%%%%%%%%%%%%%%%%%%%%%%%%%%
%Create a counter for every role here:
\newcounter{Alice}
\newcounter{Bob}


%Fill in the people in this table:
\begin{table}[H] % if you place [H] here, the table will be where you want it to be and not where LaTeX wants it to be because of free space
\begin{tabular}{p{4cm} p{5cm} p{3cm} p{2cm}}
People:& & & \\
\textbf{ALICE}, AN ARTIST & x\ref{Alice} & Master of LaTeX & John Doe \\
\textbf{BOB}, BUSDRIVER & x\ref{Bob} & Master of LaTeX & John Doe \\
\textbf{ADAM}, JUDGE & x999 & Master of LaTeX & John Doe \\
\textbf{ADAM}, JUDGE & x999 & Master of LaTeX & John Doe \\
\textbf{ADAM}, JUDGE & x999 & Master of LaTeX & John Doe \\
\textbf{ADAM}, JUDGE & x999 & Master of LaTeX & John Doe \\
\textbf{ADAM}, JUDGE & x999 & Master of LaTeX & John Doe \\
& & & \\
\multicolumn{2}{l}{Location: Huisum} &\multicolumn{2}{l}{Time: Present (1811)} \\

\end{tabular}
\end{table}

%%%%%%%%%%%%%%%%%%%%%%%%%%%%%%%

\begin{table}[H] % if you place [H] here, the table will be where you want it to be and not where LaTeX wants it to be because of free space
\begin{tabular}{|c|c|c|}
\hline
Scene & Pages & Adam \\ \hline
1 & 1 - 2 & \cellcolor{TableColorAppearance} \\ \hline
2 & 2 - 3 & \\ \hline
3 & 4 - 7 & \cellcolor{TableColorSemiAppearance} + \\ \hline
4 & 7 - 10 & \\ \hline

\end{tabular}
\end{table}

\end{flushleft}
\end{titlepage}



\begin{flushleft}

\scene{First scene - Lorem ipsum}{Alice, Bob (+ Mr. Evil)}

\BlockStageDirection \lipsum[1]

\character{Alice} Lorem ipsum dolor sit amet, consetetur sadipscing elitr, sed diam nonumy
eirmod tempor invidunt ut labore et dolore magna aliquyam erat, sed diam
voluptua. \InlineStageDirection{At vero eos et accusam et justo duo dolores et ea rebum. Stet clita kasd gubergren, no sea takimata sanctus est Lorem ipsum dolor sit
amet.} Lorem ipsum dolor sit amet, consetetur sadipscing elitr, sed diam
nonumy eirmod tempor invidunt ut labore et dolore magna aliquyam erat,
sed diam voluptua. At vero eos et accusam et justo duo dolores et ea rebum.
Stet clita kasd gubergren, no sea takimata sanctus est Lorem ipsum dolor
sit amet. Lorem ipsum dolor sit amet, consetetur sadipscing elitr, sed
diam nonumy eirmod tempor invidunt ut labore et dolore magna aliquyam
erat, sed diam voluptua. At vero eos et accusam et justo duo dolores
et ea rebum. Stet clita kasd gubergren, no sea takimata sanctus est Lorem
ipsum dolor sit amet.

\BlockStageDirection Bob enters the stage.

\character{Bob} \InlineStageDirection{quietly} \lipsum[2] 

\character{Alice} \lipsum[3]

\character{Bob} \lipsum[4]

\character{Alice} \lipsum[5]

\BlockStageDirection Alice leaves.

\scene{Second scene - dolor sit amet}{Alice, Bob (+ Mr. Evil)}

\BlockStageDirection Alice returns.

\character{Bob} \lipsum[6]

\character{Alice} \InlineStageDirection{angrily} \lipsum[7]

\character{Bob} \lipsum[8]

\character{Alice} \lipsum[9]

\character{Bob} \lipsum[10]

\end{flushleft}
\end{document}