\begin{titlepage}
  \begin{center}
  \TitlePageLineBold{18pt}{18pt}{This text belongs to:} 
  \vspace{1.25cm}
  \textcolor{SignatureLineColor}{\rule{1\textwidth}{0.8pt}}
  
  \TitlePageLine{18pt}{18pt}{Member of} 
  \TitlePageLineBold{36pt}{36pt}{\GroupName,}
  \TitlePageLine{18pt}{18pt}{Visit us on} 
  \TitlePageLineBold{36pt}{36pt}{\PerformanceDates}
  \TitlePageLine{18pt}{18pt}{at} 
  \TitlePageLineBold{36pt}{36pt}{\PerformanceLocation}
  \TitlePageLine{24pt}{36pt}{to see us perform “\PlayTitle” by \PlayAuthor.}
  \TitlePageLineBold{36pt}{36pt}{Free admission!}
  \vspace{2.5cm}
  \TitlePageLine{18pt}{27pt}{In case you find this playbook, it would be great if you could notify us at
  \setuldepth{text without dee* letters}% Prevent underline to be too deep if email adress contains deep letters (such as p or q)
  \href{mailto:group@example.com}{\ul{group@example.com}}.}
  \end{center}
  \end{titlepage}
  
  \begin{titlepage}
  \begin{flushleft}
  \TitlePageLineBold{26pt}{26pt}{\setuldepth{Der}\ul\PlayTitle}
  %\vspace{0.5cm}
  
  %Create a counter for every role here:
\newcounter{doe} \regtotcounter{doe}
\newcounter{smith} \regtotcounter{smith}
\newcounter{jack} \regtotcounter{jack}

  
  % Fill in the people in this table. The first column is for the character's name and description.
  % It can be helpful to have their "nickname" that is used for cues (= counters) to be set in bold font.
  \begin{table}[H] % if you place [H] here, the table will be where you want it to be and not where LaTeX wants it to be because of free space
  \begin{tabular}{p{13.5cm} p{0.5cm} p{0.2cm} p{2cm}}
  \multicolumn{1}{l}{Characters: \hspace{10cm} In: Germany} &\multicolumn{2}{l}{Time: Present (2024)}\\
  JOHN \textbf{DOE}, Some amazing person & x\total{doe} & Actor 1\\
  JANE \textbf{SMITH}, Another amazing person & x\total{smith} & Actress 2\\
  \textbf{JACK} WHOEVER, Not so relevant & x\total{jack} & Actress 3\\
  \end{tabular}
  \end{table}
  
  
  \setcounter{tocdepth}{2}
  \normalsize
  \pdfbookmark[1]{\contentsname}{toc}\tableofcontents 	%Nimmt das Inhaltsverzeichnis in die PDF Lesezeichenstruktur auf und erzeugt das Inhaltsverzeichnis.
  
  
  
  %%%%%%%%%%%%%%%%%%%%%%%%%%%%%%%
  \begin{landscape}
  
  \begin{table}[H] % if you place [H] here, the table will be where you want it to be and not where LaTeX wants it to be because of free space
  \begin{tabular}{|c|c|c|c|c|c|c|c|c|c|c|c|c|c|c|c|c|}
  \hline
  Scene & Pages & Doe & Smith & Jack \\ \hline
  % For semi-appearances, use \cellcolor{TableColorSemiAppearance} +
  1.1 & ? - ? & \cellcolor{TableColorAppearance} &  & \cellcolor{TableColorSemiAppearance} +\\ \hline
  1.2 & ? - ? & \cellcolor{TableColorAppearance} &  & \\ \hline
  2.1 & ? - ? & \cellcolor{TableColorSemiAppearance} + & \cellcolor{TableColorAppearance} & \\ \hline
  2.2 & ? - ? & \cellcolor{TableColorAppearance} &  & \\ \hline
  2.3 & ? - ? & \cellcolor{TableColorAppearance} & \cellcolor{TableColorAppearance} & \cellcolor{TableColorAppearance}\\ \hline
  \end{tabular}
  \end{table}
  
  \end{landscape}
  
  \end{flushleft}
\end{titlepage}
  